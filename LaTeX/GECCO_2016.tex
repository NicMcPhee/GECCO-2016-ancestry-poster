% Research Paper for GECCO 2015
% by Nic McPhee, Kirbie Dramdahl, and David Donatucci

\documentclass{sig-alternate}

%\usepackage{parskip}
%\usepackage{times} %For typeface
%\usepackage{graphicx}
%\usepackage{algorithm}
%\usepackage{algorithm,algorithmic}
%\usepackage[justification=centering]{caption}[2007/12/23]
%\usepackage{url}
\sloppy

\setlength{\parindent}{0.5cm} 

\newcommand{\citep}[1]{\cite{#1}}

%\DeclareGraphicsRule{.tif}{png}{.png}{`convert #1 `dirname #1`/`basename #1 .tif`.png}

\begin{document}

\conferenceinfo{GECCO'16,} {July 20-24, 2016, Denver, CO, USA.}
\CopyrightYear{2016}
\crdata{TBA}
\clubpenalty=10000
\widowpenalty = 10000
    
\title{Visualizing genetic programming ancestries using graph databases}

\numberofauthors{4}
\author{
\alignauthor
Nicholas Freitag McPhee\\
	\affaddr{Division of Science and Mathematics}\\
	\affaddr{University of Minnesota, Morris}\\
	\affaddr{Morris, MN USA-56267}\\
	\email{mcphee@morris.umn.edu}
\alignauthor
Maggie M. Casale\\
	\affaddr{Division of Science and Mathematics}\\
	\affaddr{University of Minnesota, Morris}\\
	\affaddr{Morris, MN USA-56267}\\
	\email{casal033@morris.umn.edu}
\alignauthor
Thomas Helmuth\\
	\affaddr{Computer Science Department}\\
	\affaddr{Washington and Lee University}\\
	\affaddr{Lexington, VA USA-24450}\\
	\email{helmutht@wlu.edu}
\alignauthor
Lee Spector\\
	\affaddr{Cognitive Science}\\
	\affaddr{Hampshire College}\\
	\affaddr{Amherst, MA USA-01002}\\
	\email{lspector@hampshire.edu}
}

\maketitle

\begin{abstract}

\emph{We are submitting this in the hopes of it being a \textbf{poster} and not a paper. There's just not a separate mechanism for submitting specifically for posters. Thanks.}

Previous work has demonstrated the utility of graph databases as a tool for collecting and analyzing ancestry in evolutionary computation runs. That work focused on sections of individual runs, whereas this poster illustrates the application of these ideas on the entirety of large runs (up to one million individuals) and combinations of multiple runs.

This will include a graph showing \emph{all} the ancestors of successful individuals from a variety of stack-based genetic programming runs on software synthesis problems. We will demonstrate how these graphs can highlight critical moments in the evolutionary process, and use them to compare the dynamics when using different evolutionary tools, such as different selection mechanisms or representations.

We will also provide examples of the scripting used to load and analyze our evolutionary data into the Titan graph database system.

\end{abstract}

%	\begin{itemize}
%	
%	\item Lexicase vs. Tournament Selection
%	\item Success vs. Failure
%	\item Tree-Based Examples
%	\item Genetic Algorithm-Based Examples
%	\item Using Titan
%	
%	\end{itemize}

\section{Introduction}
\label{sec:introduction}

We can use graph databases to store and analyze ancestry information in runs, and then use that to make super cool pictures.

\section{Examples}
\label{sec:examples}

Explain replace-space-with-newline? At the very least cite Helmuth GECCO 2015.

Mention we use Graphviz/dot to draw the pictures.

Examples: Long runs like autoconstruction, comparison of successful and unsuccessful and/or comparison of lexicase and tournament.

\begin{figure}
\centering
\includegraphics[height=\textheight]{../Figures/ancestors_RSWN_run5_2016_01_small.pdf}
\caption{Stuff}
\label{fig:autoconstructionAncestry}
\end{figure}

\section{Tools}
\label{sec:tools}

A little about Titan/gremlin/tinkerpop? URLs for these tools in footnotes.

\section{Conclusion}
\label{sec:conclusion}

\bibliographystyle{abbrv}
\bibliography{GECCO_2016}

\end{document}